\chapter{Pose estimation, elementary method}

%-----------------------------------------------------
%-----------------------------------------------------
%-----------------------------------------------------

\section{Introduction}

This chapter present the "elementary" methods, that compute the pose of an image
from observations that can, typically, be tie points (relative pose) or 
ground control point (relative or absolute pose).

By elementary we means algorithms that compute directly a solution for a limited
(typically $2$ or $3$, a few with tomasi-kanade) number of images. These algorithms
are the "tacticle" part, the result of these elementary algorithms will be used as
elementary part of the puzzle by more global method that will try to have a "strategic"  view.


%-----------------------------------------------------
%-----------------------------------------------------
%-----------------------------------------------------

\section{Space resection, calibrated case}

\label{SR_Cal}

    %-----------------------------------------------------
\subsection{Introduction}

We deal here with the following problem,  we have :

\begin{itemize}
   \item a calibrated camera ;
   \item a set of point for which we know the  $3d$ word coordinates $G_k$ and their 
        $2d$ coordinate $p_k$ in a image acquired with this camera;
\end{itemize}


We want to extract the pose $R,C$ of the camera (Rotation,center) such that for every point
we have the usual projection equation:

\begin{equation}
       \mathcal I(\pi (R*(G_k-C))) = p_k \label{EQ:PROJ}
\end{equation}


Each equation~\ref{EQ:PROJ}  creates $2$ constraints : one on line and one on column.
Also the pose $R,C$ has $6$ degrees of freedom, so we can expect that the problem has
a finite number of solution when we have $3$ points (in general less than $3$ will create
infinite number and more than $3$ no solution).

So here we will deal specifically  with the computation of the $R,C$ satisfying
exactly the equation~\ref{EQ:PROJ}  from a set of $3$ correspondance. Other
chapter will discuss of how we use more (possibly many) correspondance for beign 
robust to outlier (for
example with Ransac) or beign more accurate in presence of "gaussian" noise (for 
example with least-square like) .


This problem occurs in two case in photogrammetric pipeline :

\begin{itemize}
   \item the first case is in fact rather rare, it's the case where we have "real" GCP
         an approximate calibration, and want to compute the initial pose of a camera ;
         it's rare because the requirement of having at least $3$ GCP visible is rather
         high; by the way it can occurs in calibrating a camera on a calibration fields
         with many GCP;

 \item the second case is when we have a set of $N$ already oriented camera  ($N\geq 2 $),
       we want to orientate a new camera \emph{relatively} to this set, we have a
       set of $M \geq 3$ multiple tie point , each tie point being visible in the new
       camera and least two of already oriented camera;  when it is the case,
       we can compute for each tie point, by bundle intesection, an estimation of the
       "ground" coodinate of the point in the current system, and then estimate a pose
       for the new camera coherent with the existing one;
       this case is extremely frequent in automatic photogrammetric
       pipeline (at least will be in {\tt MMVII});

\end{itemize}

The {\tt MMVII} code corresponding to this section can be found in 
{\tt PoseEstim/CalibratedSpaceResection.cpp}.

    %-----------------------------------------------------

\subsection{Putting things in equation}

           %  -  -  -  -  -  -  -  -  -  -  -  -  -  -  -
\subsubsection{Equivalence to local coordinates}

\label{SpRes:EquivLocCoord}

First, we remark that if we know the local coordinates $L_k$ of the points in the repair
of the camera, the problem becomes easy . Let show it, we will have to find a translation-rotation such that  :

\begin{equation}
       R*(G_k-C) = L_k  ; k\in{1,2,3} \label{SpResecEQ:WL}
\end{equation}


For each pair $k,k'$, we have (noting $\Vec{X}_{kk'} = \overrightarrow{X_{k}X_{k'}} $):


\begin{equation}
	R* \Vec{G}_{kk'} =  \Vec{L}_{kk'} \label{SpResecEQ:WLVECT}
\end{equation}

As  $R$ is a rotation we have $R(A \wedge  B) = R(A) \wedge   R(B) $ and then:

\begin{equation}
	R* \begin{bmatrix} \Vec{G}_{12} & \Vec{G}_{23} &  \Vec{G}_{12} \wedge \Vec{G}_{23} \end{bmatrix} 
        =  \begin{bmatrix} \Vec{L}_{12} & \Vec{L}_{23} &  \Vec{L}_{12} \wedge \Vec{L}_{23} \end{bmatrix} 
        \label{SpResecEQ:WLVECT}
\end{equation}

Equation~\ref{SpResecEQ:WLVECT} allows to compute $R$ by matrix inversion and multiplication, and 
once $R$ is known it can be injected in equation~\ref{SpResecEQ:WL} to compute $C$.

           %  -  -  -  -  -  -  -  -  -  -  -  -  -  -  -
\subsubsection{Equivalence to compute depth}

We need to compute the local coordinates . For this, we know for each point the bundle its belongs
to because we know internal calibration. Le $B_k$ be the bundle bearing the point $p_k$ We can write symbolicly :


\begin{equation}
	B_k =  \pi^{-1} (\mathcal I ^{-1} (p_k)) \label{SpResecEQ:DefBundle}
\end{equation}



Let write $B_k = (0,\Vec{u}_k)$,  we can compute $\Vec{u}_k$ with internal calibration and
as we know that $L_k \in B_k$ we can write  :


\begin{equation}
	L_k = \lambda_k \Vec{u}_k \label{SpResecEQ:DefLambda}
\end{equation}

So to solve our problem it is sufficent to estimate $\lambda_k$.

           %  -  -  -  -  -  -  -  -  -  -  -  -  -  -  -
\subsubsection{Setting equations}

Now we remember that rotation and translation are isometric, so we have conservation of
distances between  word an local coordinates. We can then write :

\begin{equation}
  D_{kk'}=||\Vec{G}_{kk'}||   = || \Vec{L}_{kk'} || = || \lambda_k \Vec{u}_k -  \lambda_{k'} \Vec{u}_{k'}|| \label{SpResecEQ:ConsDist}
\end{equation}


In equation~\ref{SpResecEQ:ConsDist}, the $D_{kk'}$ and  $\Vec{u}_k$ are knowns, so we have $3$ equation  for
$3$ unkowns, so far so good \dots 



    %-----------------------------------------------------

\subsection{Solving $3$  equations with $3$ unknowns}

\subsubsection{Fix notation an supress 1 unknown}

We change notation to have lighter formula and to make them coherent with the {\tt C++} code of {\tt MMVII}
impplemating this resolution. We call $\Vec{A} , \Vec{B}, \Vec{C}$  the bundles, $D_{AB}$ the distances, \dots
equations become :

\begin{equation}
	D_{AB} = || \lambda_a \Vec{A} -  \lambda_{b} \Vec{B} || \label{SpResecEQ:ABC}
\end{equation}

By doing quotient of equation~\ref{SpResecEQ:ABC} we have :

\begin{equation}
	\rho^A_{bc}  
	=	\frac{D^2_{AB}}{D^2_{AC}} 
	= \frac{|| \Vec{A} -  \frac{\lambda_{b}}{\lambda_{a}} \Vec{B} ||^2 }{||\Vec{A} -  \frac{\lambda_{c}}{\lambda_{a}} \Vec{C}||^2}
	\;\;
	\rho^C_{ba}  
	=	\frac{D^2_{CB}}{D^2_{CA}} 
	= \frac{|| \frac{\lambda_{c}}{\lambda_{a}} \Vec{C} -  \frac{\lambda_{b}}{\lambda_{a}} \Vec{B} ||^2 }
	       {||\Vec{A} -  \frac{\lambda_{c}}{\lambda_{a}} \Vec{C}||^2}
	       \label{SpResecEQ:Lambda}
\end{equation}

Now we see that in equation~\ref{SpResecEQ:Lambda}, only the ratio $\frac{\lambda_{b}}{\lambda_{a}}$ and $\frac{\lambda_{c}}{\lambda_{a}}$
matters, so one can solve them a system of $2$ equation with $2$ unknowns. Once we will have computed these ratio the triangle
$(\Vec{A} , \frac{\lambda_{b}}{\lambda_{a}} \Vec{B},  \frac{\lambda_{c}}{\lambda_{a}} \Vec{C})$ will be propoprtionnal the
triangle $(G_A,G_B,G_C)$ and it will be sufficient to compute this proportion.


           %  -  -  -  -  -  -  -  -  -  -  -  -  -  -  -
\subsubsection{Supression another unknown}

We fix the notation for ratio :

\begin{equation}
	\frac{\lambda_{b}}{\lambda_{a}} = 1+ b  \;\;
	\frac{\lambda_{c}}{\lambda_{a}} = 1+ c
\end{equation}


The $1$ in $1+b$, is to be coherent with {\tt C++} implementation where this make a more numerically stable system.
Also to be coherent with {\tt C++} implementation, we will make the hypothesis that bundles have been normalized :

\begin{equation}
	|| \Vec{A}|| = || \Vec{B}|| = || \Vec{C}|| = 1
\end{equation}

So equations~\ref{SpResecEQ:ABC} rewrites :

\begin{equation}
	\rho^A_{bc}  
	= \frac{|| \Vec{A} -  (1+b) \Vec{B} ||^2 }{||\Vec{A} - (1+c) \Vec{C}||^2}
	= \frac{|| \overrightarrow{AB} +  b \Vec{B} ||^2 }{||\overrightarrow{AC}  + c \Vec{C}||^2}  \label{SpResecEQ:EqFracABC}
\end{equation}
\begin{equation}
	\rho^C_{ba}  
	= \frac{|| (1+c) \Vec{C} -  (1+b) \Vec{B} ||^2 } {||\Vec{A} -  (1+c) \Vec{C}||^2}
	= \frac{|| \overrightarrow{BC} + c \Vec{C} -  b \Vec{B} ||^2 } {||\overrightarrow{AC}  + c \Vec{C}||^2}
	 \label{SpResecEQ:EqFracCBA}
\end{equation}

Now we can write equation~\ref{SpResecEQ:EqFracABC} :

\begin{equation}
	b^2 + 2 b \overrightarrow{AB}. \Vec{B} + (||AB||^2 -   \rho^A_{bc} ||\overrightarrow{AC}  + c \Vec{C}||^2)
	\label{SpResecEQ:Pol2B}
\end{equation}

Equation~\ref{SpResecEQ:Pol2B} can be see as $2d$ degree polynom in $b$, and we can express $b$ as function of $c$:

\begin{equation}
	b = -\overrightarrow{AB}. \Vec{B} \pm \sqrt{Q(c)}  = -\overrightarrow{AB}. \Vec{B} + \epsilon \sqrt{Q(c)} 
	 \;\;\;  \epsilon \in \{-1,1\}  \label{SpResecEQ:SolEqD2}
\end{equation}

With $Q(c)$ being a $2$ degree polynom in $c$:
\begin{equation}
	Q(c) = (\overrightarrow{AB}. \Vec{B})^2 -  (||AB||^2 -   \rho^A_{bc} ||\overrightarrow{AC}  + c \Vec{C}||^2)
\end{equation}

           %  -  -  -  -  -  -  -  -  -  -  -  -  -  -  -
\subsubsection{Resolving last unknown}

We can now use equation~\ref{SpResecEQ:SolEqD2} to substituate $b$  in equation~\ref{SpResecEQ:SolEqD2}

\begin{equation}
	\rho^C_{ba}   ||\overrightarrow{AC}  + c \Vec{C}||^2
	= || \overrightarrow{BC} + c \Vec{C} -  (-\overrightarrow{AB}. \Vec{B} + \epsilon \sqrt{Q(c)}) \Vec{B} ||^2 
	\label{SpResecEQ:EqcInit}
\end{equation}

We note $R(c)$ the $2d$ degree polynom in $c$ defined by :

\begin{equation}
   R(c) =   \rho^C_{ba}   ||\overrightarrow{AC}  + c \Vec{C}||^2
        - Q(c)
        - ||\overrightarrow{BC} + c \Vec{C} + (\overrightarrow{AB}. \Vec{B}) \Vec{B} ||^2
\end{equation}

We note $L(c)$ the $1d$ degree polynom in $c$ defined by :

\begin{equation}
	L(c)=   ( \overrightarrow{BC}.\Vec{B} + \overrightarrow{AB}. \Vec{B} +   c \Vec{C}.\Vec{B}  ) 
\end{equation}

Then equation~\ref{SpResecEQ:EqcInit} writes :

\begin{equation}
	R(c) = 2 \epsilon L(c)  \sqrt{Q(c) }   \label{SpResecEQ:RL}
\end{equation}

Squaring equation~\ref{SpResecEQ:RL} we obtain a $4$ degree polynomial equation :

\begin{equation}
	R(c) ^2 - 4 L(c)^2  Q(c) = 0   \label{SpResecEQ:D4}
\end{equation}

We just need to solve equation~\ref{SpResecEQ:D4} to get possible values of $c$, 
then get values of $b$ using~\ref{SpResecEQ:SolEqD2}, 
then get depth $\lambda _k$, then get local coordinates $L_k$ using~\ref{SpResecEQ:DefLambda} 
and~\ref{SpResecEQ:DefBundle}, and finally finally compute the pose $R,C$  going back to~\ref{SpRes:EquivLocCoord}.

%-----------------------------------------------------
%-----------------------------------------------------
%-----------------------------------------------------

\section{Space resection, uncalibrated case}

    %-----------------------------------------------------

\subsection{Introduction}

We deal here with the following problem,  we have :

\begin{itemize}
   \item an \emph{un-calibrated} camera ;
   \item a set of point for which we know the  $3d$ word coordinates $G_k$ and their 
        $2d$ coordinate $p_k$ in a image acquired with this camera;
\end{itemize}


We want to extract \emph{simulatenously} the pose $R,C$ of the camera (Rotation,center)  and the calibration
$\mathcal I$ such that for every point we have the usual projection equation:

\begin{equation}
       \mathcal I(\pi (R*(G_k-C))) = p_k \label{EQ:PROJ}
\end{equation}

Obviously this problem require more informations ~\ref{SR_Cal}, because we want to estimate more parameters.
Also, due to correlation between internal and external parameters, the solution
are frequently not so stable. It usage is far less current than calibrated case,
to our knowledge, the two case where it appears practically are :


\begin{itemize}
   \item approximate  orientation of old images for wich we have no idea of focal length and/or principal
        point; one advantage of the method, as it restimate the principal point, is that it can work
        also with croped images (current case for historical images);

   \item for forcing conversion of  non central perspetive sensor to a central perspective sensor
        using  artficial $2d-3d$ correspondances; also theoretically not recommanded~\footnote{it's
        always better to use rigourous modelization}, it can unlock situation where you need to
        use a software that accept only central perspetive sensor; this is the case for example
        with satellite images, if one use small patches, then the projection  function can be
        approximate by a central perspetive on this patch ~\footnote{In fact, the accuracy of
        the approximation is a difficult question, depend of the size of the patch, of relief, 
        of the exact sensor \dots}.
\end{itemize}

The {\tt MMVII} code corresponding to this section can be found in 
{\tt PoseEstim/UnCalibratedSpaceResection.cpp}.
    %-----------------------------------------------------

\subsection{Hypothesis}

As we want also to estimate the distorsion, we will have to make some hypothesis
and select a model.  We will make that the distorsion is purely linear, so using
our current modelization we will set :

\begin{equation}
	\mathcal{I}  \begin{pmatrix} u \\ v \end{pmatrix}  = PP + F \begin{pmatrix} u + b_1 + b_2 v \\ v \end{pmatrix} 
\end{equation}

We remind (see \RefFantome) that there is only $2$ parameters for linear distorsion as $1$ is already include in the focal
and rotation in the plane is redundant with $3D$ rotation. 

This model is selected, not because it will be always appropriate (it will not !) but because
it is the model  for wich we can have direct solution. Sometime we will need less parameters,
for example we "know" that $b_1=0,b_2$ , or we know that $PP$ in the middle of image. Sometime
we will need more parameters, for example adding a radial distorsion. And also sometime
"more and less" \dots When we require other calibration model, we still can use this method to compute
an initial solution and then make a bundle adjustement to initialize the othe parameters.

    %-----------------------------------------------------

\subsection{Setting equations}

We write $A \sim B$ the relation indicating that $A$ and $B$ are colinear :

\begin{equation}
	  A \sim B   \Leftrightarrow  \exists \lambda : A =  \lambda B
\end{equation}

This projective relation, is related to image formula via :

\begin{equation}
	 \begin{pmatrix} u \\ v \end{pmatrix}  = \pi_0  \begin{pmatrix} x \\ y \\z  \end{pmatrix}
   \Leftrightarrow   \begin{pmatrix} u \\ v \\ 1 \end{pmatrix}  \sim  \begin{pmatrix} x \\ y \\z  \end{pmatrix}
\end{equation}

We have, noting $\mathcal{C}_{al}$ the calibration matrix :

\begin{equation}
	   \begin{pmatrix} i \\ j \\ 1\end{pmatrix}
      =   \begin{pmatrix} PP_x + F(u+p_1u+p_2v) \\ PP_y + F v \\ 1\end{pmatrix}
      =  \begin{pmatrix} F(1+p_1) & F p_2 & PP_x ) \\  0 &   F & PP_x \\  0 & 0 &1\end{pmatrix} * \begin{pmatrix} u \\  v \\ 1\end{pmatrix} 
	      =  \mathcal{C}_{al} \begin{pmatrix} u \\  v \\ 1\end{pmatrix} 
\end{equation}

We have also :

\begin{equation}
	 \begin{pmatrix} u \\  v \\ 1\end{pmatrix}
		 \sim P_c =  \trans R(P-C)
\end{equation}

So :
\begin{equation}
	   \begin{pmatrix} i \\ j \\ 1\end{pmatrix}
		   \sim \mathcal{C}_{al}  \trans R  (P-C)
\end{equation}

Noting $  \mathcal{M} =  \mathcal{C}_{al}  \trans R $, and $Tr = -\mathcal{M} C$, we have :

\begin{equation}
	\begin{pmatrix} i \\ j \end{pmatrix} = \pi_0 ( \mathcal{M} P + Tr )
\end{equation}

Noting :

\begin{equation}
	\mathcal{M}   =  \begin{pmatrix} a & b & c \\ d & e & f \\ g & h & i \end{pmatrix} 
   \;\;\; ; \;\;\;
	 Tr  =  \begin{pmatrix} t_x \\ t_y \\ t_z \end{pmatrix} 
\end{equation}

We finnaly have ;

\begin{equation}
	i = \frac{ax+by+cz+t_x}{gx+hy+iz+t_z}
   \;\;\; ; \;\;\;
	j = \frac{dx+ey+fz+t_y}{gx+hy+iz+t_z}
	\label{EqHomSRU}
\end{equation}

    %-----------------------------------------------------

\subsection{Solving equation}

    %-----------------------------------------------------
\subsubsection{General approach}

We want to compute $\mathcal{C}_{al},R,C$  from a set of correspondance $(i,j) (x,y,z)$.
We will proceed in $2$ steps :

\begin{itemize}
     \item estimate $a,b\dots ,t_x \dots$, i.e. $\mathcal{M}$ and $Tr$ using equation \ref{EqHomSRU};
     \item extract  $\mathcal{C}_{al},R,C$ from $\mathcal{M}$ and $Tr$;
\end{itemize}


    %-----------------------------------------------------
\subsubsection{Homography estimation}

Equation~\ref{EqHomSRU} is a classical equation for computing unknown homography.
Considering that $i,j,x,y,z$ are observations and $a,b\dots ,t_x \dots$ are unknowns, we can write it 
as :

\begin{equation}
	0 = {ax+by+cz+t_x} -i (gx+hy+iz+t_z)
   \;\;\; ; \;\;\;
	0 = dx+ey+fz+t_y - j (gx+hy+iz+t_z)
	\label{EqLineSRU}
\end{equation}

The nice point is that equation~\ref{EqLineSRU} is linear, easy to solve and with
many observations, we can use least square approach. The bad point is that,
as there is no constant occuring in it: 

\begin{itemize}
    \item the system is ambiguous , if $S$ is a solution $\lambda S$ is a solution
    \item worst, the null vector is a perfect solution that will always annulate the residual
	    of the least-square system.
\end{itemize}

There is different way to overcome this problem. We indicate the way used in {\tt MMVII}.
The trick, as equations ~\ref{EqLineSRU} are defined up to a scaling factor is to fix the
ambiguity by fixing arbitrary one of the variable to a non null value, for example $a=1$ or $t_z=1$. 
But which of the variable must we fix ?  If we make a bad choice, will it have an influence on the
quality.  The method we use is in fact to not really make a choice! We test all the
possible variable, each test lead to a solution and we use equation~\ref{EqHomSRU} to
select the best solution. Note, that it is fast, as the normal matrix is computed only once.
For more detail on the implementation please refer to the code which is densely
documented.


    %-----------------------------------------------------
\subsubsection{Un-mixing parameters}

Sometime, it will be sufficient to know $\mathcal{M}$ and $Tr$, because
with them we can project any point in the image using equation~\ref{EqHomSRU}.
But in other case, we will need to recover the "physical" parameters.

Considering the center, it is easy, from $Tr = -\mathcal{M} C$, we have
obviously :

\begin{equation}
     C= - \mathcal{M}^{-1} Tr
\end{equation}




% (I)    (PPx + F (u  p1 u + p2 v))     (F(1+p1)   p2F  PPx) (u)     (a b c) (u)      (u) [EqCal]
% (J) ~  (PPy + F v               )  =  (0         F    PPy) (v)  =  (0 e f) (v) =  C (v)
% (1)    (                       1)     (0         0     1)  (1)     (0 0 1) (1)      (1)


%-----------------------------------------------------
%-----------------------------------------------------
%-----------------------------------------------------

\section{Ortographic case with Tomasi-Kanabe}

%-----------------------------------------------------
%-----------------------------------------------------
%-----------------------------------------------------

\section{$2$-images orientation with essentiel matrix}

%-----------------------------------------------------
%-----------------------------------------------------
%-----------------------------------------------------

\section{$2$-images, case planary scenes}


    %-----------------------------------------------------

\subsection{Notes for TD}


\begin{verbatim}
TD1 : SPACE RESECTION

*  Begin with a calibrated camera w/o distortion,
*  From $3$ point compute 2 lamba with MMVII , then generate the rotations, solve ambiquity (with $4$ points)
*  From data with outlier (50\% on 50 point) use ransac like to make a robust init.

*  Do the same thing with distorsion => inverse disorsion (iterative method ? Lesqt Sq ? Majical optimzer of pytthon ?)

*  eventually => interface


TD2 :  CMP CALIB

TD3 : CALIB CONV
\end{verbatim}




